% Options for packages loaded elsewhere
\PassOptionsToPackage{unicode}{hyperref}
\PassOptionsToPackage{hyphens}{url}
%
\documentclass[
]{article}
\usepackage{amsmath,amssymb}
\usepackage{iftex}
\ifPDFTeX
  \usepackage[T1]{fontenc}
  \usepackage[utf8]{inputenc}
  \usepackage{textcomp} % provide euro and other symbols
\else % if luatex or xetex
  \usepackage{unicode-math} % this also loads fontspec
  \defaultfontfeatures{Scale=MatchLowercase}
  \defaultfontfeatures[\rmfamily]{Ligatures=TeX,Scale=1}
\fi
\usepackage{lmodern}
\ifPDFTeX\else
  % xetex/luatex font selection
\fi
% Use upquote if available, for straight quotes in verbatim environments
\IfFileExists{upquote.sty}{\usepackage{upquote}}{}
\IfFileExists{microtype.sty}{% use microtype if available
  \usepackage[]{microtype}
  \UseMicrotypeSet[protrusion]{basicmath} % disable protrusion for tt fonts
}{}
\makeatletter
\@ifundefined{KOMAClassName}{% if non-KOMA class
  \IfFileExists{parskip.sty}{%
    \usepackage{parskip}
  }{% else
    \setlength{\parindent}{0pt}
    \setlength{\parskip}{6pt plus 2pt minus 1pt}}
}{% if KOMA class
  \KOMAoptions{parskip=half}}
\makeatother
\usepackage{xcolor}
\usepackage[margin=1in]{geometry}
\usepackage{color}
\usepackage{fancyvrb}
\newcommand{\VerbBar}{|}
\newcommand{\VERB}{\Verb[commandchars=\\\{\}]}
\DefineVerbatimEnvironment{Highlighting}{Verbatim}{commandchars=\\\{\}}
% Add ',fontsize=\small' for more characters per line
\usepackage{framed}
\definecolor{shadecolor}{RGB}{248,248,248}
\newenvironment{Shaded}{\begin{snugshade}}{\end{snugshade}}
\newcommand{\AlertTok}[1]{\textcolor[rgb]{0.94,0.16,0.16}{#1}}
\newcommand{\AnnotationTok}[1]{\textcolor[rgb]{0.56,0.35,0.01}{\textbf{\textit{#1}}}}
\newcommand{\AttributeTok}[1]{\textcolor[rgb]{0.13,0.29,0.53}{#1}}
\newcommand{\BaseNTok}[1]{\textcolor[rgb]{0.00,0.00,0.81}{#1}}
\newcommand{\BuiltInTok}[1]{#1}
\newcommand{\CharTok}[1]{\textcolor[rgb]{0.31,0.60,0.02}{#1}}
\newcommand{\CommentTok}[1]{\textcolor[rgb]{0.56,0.35,0.01}{\textit{#1}}}
\newcommand{\CommentVarTok}[1]{\textcolor[rgb]{0.56,0.35,0.01}{\textbf{\textit{#1}}}}
\newcommand{\ConstantTok}[1]{\textcolor[rgb]{0.56,0.35,0.01}{#1}}
\newcommand{\ControlFlowTok}[1]{\textcolor[rgb]{0.13,0.29,0.53}{\textbf{#1}}}
\newcommand{\DataTypeTok}[1]{\textcolor[rgb]{0.13,0.29,0.53}{#1}}
\newcommand{\DecValTok}[1]{\textcolor[rgb]{0.00,0.00,0.81}{#1}}
\newcommand{\DocumentationTok}[1]{\textcolor[rgb]{0.56,0.35,0.01}{\textbf{\textit{#1}}}}
\newcommand{\ErrorTok}[1]{\textcolor[rgb]{0.64,0.00,0.00}{\textbf{#1}}}
\newcommand{\ExtensionTok}[1]{#1}
\newcommand{\FloatTok}[1]{\textcolor[rgb]{0.00,0.00,0.81}{#1}}
\newcommand{\FunctionTok}[1]{\textcolor[rgb]{0.13,0.29,0.53}{\textbf{#1}}}
\newcommand{\ImportTok}[1]{#1}
\newcommand{\InformationTok}[1]{\textcolor[rgb]{0.56,0.35,0.01}{\textbf{\textit{#1}}}}
\newcommand{\KeywordTok}[1]{\textcolor[rgb]{0.13,0.29,0.53}{\textbf{#1}}}
\newcommand{\NormalTok}[1]{#1}
\newcommand{\OperatorTok}[1]{\textcolor[rgb]{0.81,0.36,0.00}{\textbf{#1}}}
\newcommand{\OtherTok}[1]{\textcolor[rgb]{0.56,0.35,0.01}{#1}}
\newcommand{\PreprocessorTok}[1]{\textcolor[rgb]{0.56,0.35,0.01}{\textit{#1}}}
\newcommand{\RegionMarkerTok}[1]{#1}
\newcommand{\SpecialCharTok}[1]{\textcolor[rgb]{0.81,0.36,0.00}{\textbf{#1}}}
\newcommand{\SpecialStringTok}[1]{\textcolor[rgb]{0.31,0.60,0.02}{#1}}
\newcommand{\StringTok}[1]{\textcolor[rgb]{0.31,0.60,0.02}{#1}}
\newcommand{\VariableTok}[1]{\textcolor[rgb]{0.00,0.00,0.00}{#1}}
\newcommand{\VerbatimStringTok}[1]{\textcolor[rgb]{0.31,0.60,0.02}{#1}}
\newcommand{\WarningTok}[1]{\textcolor[rgb]{0.56,0.35,0.01}{\textbf{\textit{#1}}}}
\usepackage{longtable,booktabs,array}
\usepackage{calc} % for calculating minipage widths
% Correct order of tables after \paragraph or \subparagraph
\usepackage{etoolbox}
\makeatletter
\patchcmd\longtable{\par}{\if@noskipsec\mbox{}\fi\par}{}{}
\makeatother
% Allow footnotes in longtable head/foot
\IfFileExists{footnotehyper.sty}{\usepackage{footnotehyper}}{\usepackage{footnote}}
\makesavenoteenv{longtable}
\usepackage{graphicx}
\makeatletter
\def\maxwidth{\ifdim\Gin@nat@width>\linewidth\linewidth\else\Gin@nat@width\fi}
\def\maxheight{\ifdim\Gin@nat@height>\textheight\textheight\else\Gin@nat@height\fi}
\makeatother
% Scale images if necessary, so that they will not overflow the page
% margins by default, and it is still possible to overwrite the defaults
% using explicit options in \includegraphics[width, height, ...]{}
\setkeys{Gin}{width=\maxwidth,height=\maxheight,keepaspectratio}
% Set default figure placement to htbp
\makeatletter
\def\fps@figure{htbp}
\makeatother
\setlength{\emergencystretch}{3em} % prevent overfull lines
\providecommand{\tightlist}{%
  \setlength{\itemsep}{0pt}\setlength{\parskip}{0pt}}
\setcounter{secnumdepth}{-\maxdimen} % remove section numbering
\ifLuaTeX
  \usepackage{selnolig}  % disable illegal ligatures
\fi
\usepackage{bookmark}
\IfFileExists{xurl.sty}{\usepackage{xurl}}{} % add URL line breaks if available
\urlstyle{same}
\hypersetup{
  pdftitle={Curso de Especialização em Data Science e Estatística Aplicada Módulo I - Introdução à programação Atividade Avaliativa},
  pdfauthor={Prof.~Dr.~Fabrizzio Soares},
  hidelinks,
  pdfcreator={LaTeX via pandoc}}

\title{Curso de Especialização em Data Science e Estatística Aplicada
Módulo I - Introdução à programação\\
Atividade Avaliativa}
\author{Prof.~Dr.~Fabrizzio Soares}
\date{2024-05-10}

\begin{document}
\maketitle

\section{Instruções}\label{instruuxe7uxf5es}

\begin{itemize}
\item
  O desenvolvimento desta atividade deve ser realizada de forma
  individual.
\item
  Deve-se completar o arquivo Rmd enviado na atividade.
\item
  É necessário devolver o arquivo em Rmd e em pdf.
\item
  Valor da atividade: 10 pontos.
\end{itemize}

\textbf{Data disponível: 09/05/2024}

\textbf{Data máxima para entrega: 31/05/2024}

\section{Atividade}\label{atividade}

Nesta atividade avaliativa, desenvolveremos um pequeno projeto para
analisar dados do \textbf{Sinan} (\emph{Sistema de Informação de Agravos
de Notificação}) do Ministério da Saúde.

\subsection{Sobre o Sinan}\label{sobre-o-sinan}

O Sistema de Informação de Agravos de Notificação (Sinan) é alimentado,
pela notificação e investigação de casos de doenças e agravos que
constam da lista nacional de doenças de notificação compulsória
(Portaria GM/MS no. 217, de 1 de março de 2023), mas é facultado a
estados e municípios incluir outros problemas de saúde importantes em
sua região. Sua utilização efetiva permite a realização do diagnóstico
da ocorrência de um evento na população, podendo fornecer subsídios para
explicações dos agravos de notificação compulsória, além de indicar
riscos aos quais as pessoas estão sujeitas, contribuindo assim, para a
identificação da realidade epidemiológica de determinada área
geográfica. O seu uso sistemático, de forma descentralizada, contribui
para a democratização da informação, permitindo que todos os
profissionais de saúde tenham acesso à informação e as tornem
disponíveis para a comunidade. É, portanto, um instrumento relevante
para auxiliar o planejamento da saúde, definir prioridades de
intervenção, além de permitir que seja avaliado o impacto das
intervenções. Fonte: \href{http://portalsinan.saude.gov.br/}{Portal
SINAN} - \url{http://portalsinan.saude.gov.br/}.

\subsection{Sobre os dados}\label{sobre-os-dados}

Para esta atividade, foram coletados dados de Febre de Chikungunya dos
estados de Goiás, do Maranhão e do Brasil.

Os dados podem ser baixados da página do
\href{https://datasus.saude.gov.br/acesso-a-informacao/doencas-e-agravos-de-notificacao-de-2007-em-diante-sinan/}{SINAN}
\url{https://datasus.saude.gov.br/acesso-a-informacao/doencas-e-agravos-de-notificacao-de-2007-em-diante-sinan/},
onde é possível escolher a Doenças e Agravos de Notificação, e a área
geográfica de abrangência.

\includegraphics{img/sinan1.png} \includegraphics{img/sinan2.png}

Uma vez escolhido a doença e a área geográfica de abrangência, na
próxima página são escolhidos os dados de interesse.

\includegraphics{img/sinan3.png}

Foram escolhidos para análise, como linha, o ano de notificação, e como
coluna, a faixa etária. Foram escolhidos todos os períodos disponíveis
de 2017 a 2024.

Os dados podem ser visualizados em formato com tabela com bordas e podem
ser baixados em formato CSV.

\includegraphics{img/sinan6.png}

Você pode escolher também em colunas separadas por ponto e vírgula (;).

\includegraphics{img/sinan4.png}

Com o objetivo de simplificar essa tarefa, os dados foram coletados e
foram organizados em forma de vetores.

\includegraphics{img/sinan5.png}

Foram descartadas os nomes de colunas, porém o nome das linhas (anos)
foram mantidos para auxiliar na reorganização dos dados durante a
extração. A linha e a coluna de totais também foi descartada.

As colunas dos dados são apresentados na Tabela 1.

\newpage

\begin{longtable}[]{@{}ll@{}}
\caption{Faixas de idades.}\tabularnewline
\toprule\noalign{}
Coluna & Descrição \\
\midrule\noalign{}
\endfirsthead
\toprule\noalign{}
Coluna & Descrição \\
\midrule\noalign{}
\endhead
\bottomrule\noalign{}
\endlastfoot
1 & ``Ano notificação'' \\
2 & ``Em branco/IGN'' \\
3 & ``\textless1 Ano'' \\
4 & ``1-4'' \\
5 & ``5-9'' \\
6 & ``10-14'' \\
7 & ``15-19'' \\
8 & ``20-39'' \\
9 & ``40-59'' \\
10 & ``60-64'' \\
11 & ``65-69'' \\
12 & ``70-79'' \\
13 & ``80 e +'' \\
\end{longtable}

A seguir, estão os dados de Goiás,

\begin{Shaded}
\begin{Highlighting}[]
\NormalTok{dados\_goias }\OtherTok{\textless{}{-}} \FunctionTok{c}\NormalTok{(}\DecValTok{2017}\NormalTok{,}\ConstantTok{NA}\NormalTok{,}\DecValTok{23}\NormalTok{,}\DecValTok{15}\NormalTok{,}\DecValTok{27}\NormalTok{,}\DecValTok{35}\NormalTok{,}\DecValTok{55}\NormalTok{,}
                \DecValTok{269}\NormalTok{,}\DecValTok{238}\NormalTok{,}\DecValTok{29}\NormalTok{,}\DecValTok{17}\NormalTok{,}\DecValTok{18}\NormalTok{,}\DecValTok{5}\NormalTok{,}
               \DecValTok{2018}\NormalTok{,}\ConstantTok{NA}\NormalTok{,}\DecValTok{8}\NormalTok{,}\DecValTok{6}\NormalTok{,}\DecValTok{24}\NormalTok{,}\DecValTok{30}\NormalTok{,}\DecValTok{46}\NormalTok{,}
                \DecValTok{221}\NormalTok{,}\DecValTok{165}\NormalTok{,}\DecValTok{19}\NormalTok{,}\DecValTok{12}\NormalTok{,}\DecValTok{12}\NormalTok{,}\DecValTok{3}\NormalTok{,}
               \DecValTok{2019}\NormalTok{,}\ConstantTok{NA}\NormalTok{,}\DecValTok{4}\NormalTok{,}\DecValTok{11}\NormalTok{,}\DecValTok{14}\NormalTok{,}\DecValTok{30}\NormalTok{,}\DecValTok{29}\NormalTok{,}
                \DecValTok{148}\NormalTok{,}\DecValTok{106}\NormalTok{,}\DecValTok{13}\NormalTok{,}\DecValTok{7}\NormalTok{,}\DecValTok{17}\NormalTok{,}\DecValTok{3}\NormalTok{,}
               \DecValTok{2020}\NormalTok{,}\ConstantTok{NA}\NormalTok{,}\DecValTok{18}\NormalTok{,}\DecValTok{6}\NormalTok{,}\DecValTok{16}\NormalTok{,}\DecValTok{11}\NormalTok{,}\DecValTok{16}\NormalTok{,}
                \DecValTok{115}\NormalTok{,}\DecValTok{71}\NormalTok{,}\DecValTok{13}\NormalTok{,}\DecValTok{6}\NormalTok{,}\DecValTok{5}\NormalTok{,}\DecValTok{3}\NormalTok{,}
               \DecValTok{2021}\NormalTok{,}\DecValTok{3}\NormalTok{,}\DecValTok{22}\NormalTok{,}\DecValTok{15}\NormalTok{,}\DecValTok{32}\NormalTok{,}\DecValTok{38}\NormalTok{,}\DecValTok{51}\NormalTok{,}
                \DecValTok{367}\NormalTok{,}\DecValTok{372}\NormalTok{,}\DecValTok{50}\NormalTok{,}\DecValTok{34}\NormalTok{,}\DecValTok{37}\NormalTok{,}\DecValTok{14}\NormalTok{,}
               \DecValTok{2022}\NormalTok{,}\ConstantTok{NA}\NormalTok{,}\DecValTok{60}\NormalTok{,}\DecValTok{107}\NormalTok{,}\DecValTok{180}\NormalTok{,}\DecValTok{258}\NormalTok{,}\DecValTok{289}\NormalTok{,}
                \DecValTok{1929}\NormalTok{,}\DecValTok{2476}\NormalTok{,}\DecValTok{389}\NormalTok{,}\DecValTok{332}\NormalTok{,}\DecValTok{342}\NormalTok{,}\DecValTok{101}\NormalTok{,}
               \DecValTok{2023}\NormalTok{,}\DecValTok{1}\NormalTok{,}\DecValTok{35}\NormalTok{,}\DecValTok{80}\NormalTok{,}\DecValTok{130}\NormalTok{,}\DecValTok{187}\NormalTok{,}\DecValTok{269}\NormalTok{,}
                \DecValTok{1445}\NormalTok{,}\DecValTok{1369}\NormalTok{,}\DecValTok{222}\NormalTok{,}\DecValTok{170}\NormalTok{,}\DecValTok{181}\NormalTok{,}\DecValTok{61}\NormalTok{,}
               \DecValTok{2024}\NormalTok{,}\DecValTok{1}\NormalTok{,}\DecValTok{43}\NormalTok{,}\DecValTok{125}\NormalTok{,}\DecValTok{240}\NormalTok{,}\DecValTok{328}\NormalTok{,}\DecValTok{491}\NormalTok{,}
                \DecValTok{2477}\NormalTok{,}\DecValTok{2799}\NormalTok{,}\DecValTok{508}\NormalTok{,}\DecValTok{429}\NormalTok{,}\DecValTok{535}\NormalTok{,}\DecValTok{244}\NormalTok{)}
\end{Highlighting}
\end{Shaded}

do Maranhão

\begin{Shaded}
\begin{Highlighting}[]
\NormalTok{dados\_maranhao }\OtherTok{\textless{}{-}} \FunctionTok{c}\NormalTok{(}\DecValTok{2017}\NormalTok{,}\ConstantTok{NA}\NormalTok{,}\DecValTok{159}\NormalTok{,}\DecValTok{179}\NormalTok{,}\DecValTok{402}\NormalTok{,}\DecValTok{581}\NormalTok{,}\DecValTok{775}\NormalTok{,}
                      \DecValTok{3063}\NormalTok{,}\DecValTok{1863}\NormalTok{,}\DecValTok{305}\NormalTok{,}\DecValTok{250}\NormalTok{,}\DecValTok{305}\NormalTok{,}\DecValTok{117}\NormalTok{,}
                    \DecValTok{2018}\NormalTok{,}\ConstantTok{NA}\NormalTok{,}\DecValTok{44}\NormalTok{,}\DecValTok{30}\NormalTok{,}\DecValTok{37}\NormalTok{,}\DecValTok{40}\NormalTok{,}\DecValTok{84}\NormalTok{,}
                      \DecValTok{365}\NormalTok{,}\DecValTok{251}\NormalTok{,}\DecValTok{32}\NormalTok{,}\DecValTok{25}\NormalTok{,}\DecValTok{51}\NormalTok{,}\DecValTok{15}\NormalTok{,}
                    \DecValTok{2019}\NormalTok{,}\ConstantTok{NA}\NormalTok{,}\DecValTok{33}\NormalTok{,}\DecValTok{66}\NormalTok{,}\DecValTok{79}\NormalTok{,}\DecValTok{112}\NormalTok{,}\DecValTok{88}\NormalTok{,}
                      \DecValTok{360}\NormalTok{,}\DecValTok{186}\NormalTok{,}\DecValTok{24}\NormalTok{,}\DecValTok{16}\NormalTok{,}\DecValTok{21}\NormalTok{,}\DecValTok{4}\NormalTok{,}
                    \DecValTok{2020}\NormalTok{,}\ConstantTok{NA}\NormalTok{,}\DecValTok{26}\NormalTok{,}\DecValTok{18}\NormalTok{,}\DecValTok{28}\NormalTok{,}\DecValTok{27}\NormalTok{,}\DecValTok{23}\NormalTok{,}
                      \DecValTok{75}\NormalTok{,}\DecValTok{40}\NormalTok{,}\DecValTok{6}\NormalTok{,}\DecValTok{2}\NormalTok{,}\DecValTok{5}\NormalTok{,}\DecValTok{2}\NormalTok{,}
                    \DecValTok{2021}\NormalTok{,}\DecValTok{1}\NormalTok{,}\DecValTok{28}\NormalTok{,}\DecValTok{33}\NormalTok{,}\DecValTok{29}\NormalTok{,}\DecValTok{20}\NormalTok{,}\DecValTok{19}\NormalTok{,}
                      \DecValTok{69}\NormalTok{,}\DecValTok{55}\NormalTok{,}\DecValTok{8}\NormalTok{,}\DecValTok{9}\NormalTok{,}\DecValTok{5}\NormalTok{,}\DecValTok{3}\NormalTok{,}
                    \DecValTok{2022}\NormalTok{,}\DecValTok{5}\NormalTok{,}\DecValTok{154}\NormalTok{,}\DecValTok{144}\NormalTok{,}\DecValTok{245}\NormalTok{,}\DecValTok{275}\NormalTok{,}\DecValTok{235}\NormalTok{,}
                      \DecValTok{1183}\NormalTok{,}\DecValTok{915}\NormalTok{,}\DecValTok{129}\NormalTok{,}\DecValTok{96}\NormalTok{,}\DecValTok{105}\NormalTok{,}\DecValTok{32}\NormalTok{,}
                    \DecValTok{2023}\NormalTok{,}\DecValTok{1}\NormalTok{,}\DecValTok{78}\NormalTok{,}\DecValTok{120}\NormalTok{,}\DecValTok{267}\NormalTok{,}\DecValTok{302}\NormalTok{,}\DecValTok{260}\NormalTok{,}
                      \DecValTok{1453}\NormalTok{,}\DecValTok{1266}\NormalTok{,}\DecValTok{207}\NormalTok{,}\DecValTok{159}\NormalTok{,}\DecValTok{203}\NormalTok{,}\DecValTok{81}\NormalTok{,}
                    \DecValTok{2024}\NormalTok{,}\ConstantTok{NA}\NormalTok{,}\DecValTok{28}\NormalTok{,}\DecValTok{75}\NormalTok{,}\DecValTok{89}\NormalTok{,}\DecValTok{85}\NormalTok{,}\DecValTok{101}\NormalTok{,}
                      \DecValTok{480}\NormalTok{,}\DecValTok{335}\NormalTok{,}\DecValTok{51}\NormalTok{,}\DecValTok{39}\NormalTok{,}\DecValTok{43}\NormalTok{,}\DecValTok{15}\NormalTok{)}
\end{Highlighting}
\end{Shaded}

e do Brasil

\begin{Shaded}
\begin{Highlighting}[]
\NormalTok{dados\_brasil }\OtherTok{\textless{}{-}} \FunctionTok{c}\NormalTok{(}\DecValTok{2017}\NormalTok{,}\DecValTok{66}\NormalTok{,}\DecValTok{3346}\NormalTok{,}\DecValTok{4976}\NormalTok{,}\DecValTok{9930}\NormalTok{,}\DecValTok{13999}\NormalTok{,}\DecValTok{18984}\NormalTok{,}
                    \DecValTok{91150}\NormalTok{,}\DecValTok{69186}\NormalTok{,}\DecValTok{11401}\NormalTok{,}\DecValTok{9123}\NormalTok{,}\DecValTok{10826}\NormalTok{,}\DecValTok{4704}\NormalTok{,}
                  \DecValTok{2018}\NormalTok{,}\DecValTok{27}\NormalTok{,}\DecValTok{1715}\NormalTok{,}\DecValTok{2514}\NormalTok{,}\DecValTok{4376}\NormalTok{,}\DecValTok{6060}\NormalTok{,}\DecValTok{8301}\NormalTok{,}
                    \DecValTok{41701}\NormalTok{,}\DecValTok{36162}\NormalTok{,}\DecValTok{6306}\NormalTok{,}\DecValTok{4705}\NormalTok{,}\DecValTok{5176}\NormalTok{,}\DecValTok{1722}\NormalTok{,}
                  \DecValTok{2019}\NormalTok{,}\DecValTok{91}\NormalTok{,}\DecValTok{2226}\NormalTok{,}\DecValTok{3528}\NormalTok{,}\DecValTok{6741}\NormalTok{,}\DecValTok{9016}\NormalTok{,}\DecValTok{11645}\NormalTok{,}
                    \DecValTok{62033}\NormalTok{,}\DecValTok{54629}\NormalTok{,}\DecValTok{10042}\NormalTok{,}\DecValTok{7678}\NormalTok{,}\DecValTok{7975}\NormalTok{,}\DecValTok{2895}\NormalTok{,}
                  \DecValTok{2020}\NormalTok{,}\DecValTok{55}\NormalTok{,}\DecValTok{1820}\NormalTok{,}\DecValTok{1874}\NormalTok{,}\DecValTok{3757}\NormalTok{,}\DecValTok{4724}\NormalTok{,}\DecValTok{5577}\NormalTok{,}
                    \DecValTok{36982}\NormalTok{,}\DecValTok{33253}\NormalTok{,}\DecValTok{4878}\NormalTok{,}\DecValTok{3573}\NormalTok{,}\DecValTok{3972}\NormalTok{,}\DecValTok{1606}\NormalTok{,}
                  \DecValTok{2021}\NormalTok{,}\DecValTok{89}\NormalTok{,}\DecValTok{1789}\NormalTok{,}\DecValTok{3329}\NormalTok{,}\DecValTok{5554}\NormalTok{,}\DecValTok{6430}\NormalTok{,}\DecValTok{7189}\NormalTok{,}
                    \DecValTok{45081}\NormalTok{,}\DecValTok{40969}\NormalTok{,}\DecValTok{6212}\NormalTok{,}\DecValTok{4791}\NormalTok{,}\DecValTok{5482}\NormalTok{,}\DecValTok{2060}\NormalTok{,}
                  \DecValTok{2022}\NormalTok{,}\DecValTok{99}\NormalTok{,}\DecValTok{3723}\NormalTok{,}\DecValTok{6918}\NormalTok{,}\DecValTok{12126}\NormalTok{,}\DecValTok{15108}\NormalTok{,}\DecValTok{16824}\NormalTok{,}
                    \DecValTok{96867}\NormalTok{,}\DecValTok{83007}\NormalTok{,}\DecValTok{12675}\NormalTok{,}\DecValTok{9670}\NormalTok{,}\DecValTok{12032}\NormalTok{,}\DecValTok{4873}\NormalTok{,}
                  \DecValTok{2023}\NormalTok{,}\DecValTok{74}\NormalTok{,}\DecValTok{2352}\NormalTok{,}\DecValTok{5684}\NormalTok{,}\DecValTok{10986}\NormalTok{,}\DecValTok{14309}\NormalTok{,}\DecValTok{16772}\NormalTok{,}
                    \DecValTok{85526}\NormalTok{,}\DecValTok{73568}\NormalTok{,}\DecValTok{12268}\NormalTok{,}\DecValTok{9817}\NormalTok{,}\DecValTok{11524}\NormalTok{,}\DecValTok{4894}\NormalTok{,}
                  \DecValTok{2024}\NormalTok{,}\DecValTok{53}\NormalTok{,}\DecValTok{1850}\NormalTok{,}\DecValTok{5064}\NormalTok{,}\DecValTok{9715}\NormalTok{,}\DecValTok{12573}\NormalTok{,}\DecValTok{15512}\NormalTok{,}
                    \DecValTok{72253}\NormalTok{,}\DecValTok{66695}\NormalTok{,}\DecValTok{12289}\NormalTok{,}\DecValTok{9881}\NormalTok{,}\DecValTok{11574}\NormalTok{,}\DecValTok{4680}\NormalTok{)}
\end{Highlighting}
\end{Shaded}

\subsection{Desenvolva e/ou responda os itens
abaixo.}\label{desenvolva-eou-responda-os-itens-abaixo.}

Item 1. Montar as 3 matrizes dos dados de Goiás, Maranhão e Brasil,
respectivamente, em que, as colunas conterão os dados de faixa etária,
descartando-se a coluna ``Ano notificacao'', e ``Em branco/IGN''.

Item 2. Desenvolver comandos e gerar vetores de totais por ano, e por
faixa etária para cada conjunto de dado. Assim, serão gerados 6 vetores
(2 para cada área geográfica).

Item 3. Desenvolver comandos para concatenar os vetores de totais por
ano nas matrizes de dados, para que os dados passem a ter uma coluna de
totais por ano (\emph{lembra da função cbind?}).

Item 4. Desenvolver comandos para concatenar os vetores de totais por
faixa etária para formar uma matriz em que as linhas são as áreas
geográficas (\textbf{Goiás, Maranhão e Brasil}).

Item 5. Desenvolver comandos e para dar nome às linhas e colunas das
matrizes de dados (\emph{lembra das funções rownames e colnames?}), em
que as linhas e colunas representarão o ano de notificação e faixa
etária, respectivamente (\emph{Não se esqueça da coluna de totais!}).

Item 6. Agora é hora de analisar. Utilizando as operações de
vetores/matrizes (sum, colSums, rowSums, max, min, entre outros),
desenvolva um script para responder às seguintes perguntas:

\begin{itemize}
\tightlist
\item
  Qual é o percentual do Brasil para cada faixa etária em Goiás e no
  Maranhão?
\item
  Qual é a faixa etária com o maior número de casos em Goiás, Maranhão e
  no Brasil?
\item
  Qual é o ano com o maior número casos em Goiás, Maranhão e no Brasil?
\end{itemize}

Estes resultados podem ser exibidos de forma individual ou organizados
em forma de tabela.

\end{document}
